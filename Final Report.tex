%Document Class - Article

\documentclass[14pt]{report}

%Preamble

\title{Universal Interpreter}
\date{10/03/2020}
\author{Jerry Allan Akshay, Shiva Pundir, Rahul R and Sai Kumar}

\usepackage{indentfirst}
\usepackage{array}
\usepackage[utf8]{inputenc}
\usepackage{graphicx}
\usepackage{geometry}
\usepackage{extsizes}
\usepackage{fancyhdr}
\usepackage{titleref}
\graphicspath{ {./images/} }
\usepackage [english]{babel}
\usepackage [autostyle, english = american]{csquotes}
\MakeOuterQuote{"}

\newcommand{\changefont}{%
    \fontsize{10}{10}\selectfont
}

%Page Margins and Dimensions
\geometry{		
	a4paper,
	total={170mm,257mm},
	left=25mm,
	top=25mm,
	right=25mm,
	bottom=25mm,
	}

%Header
\pagestyle{fancy}
\fancyhf{}
\rhead{Universal Interpreter}

\renewcommand{\contentsname}{Table of Contents}

\renewcommand{\headrulewidth}{2pt}
\renewcommand{\footrulewidth}{0.4pt}



\begin{document} %Document Begins

	%Title Page

 	\maketitle
	\pagenumbering{gobble} %Disable Page Numbering

	\newpage

	%Acknowledgement

	\chapter*{Acknowledgement}\label{chapter}
		
		\lhead{\titleref{chapter}}

		The satisfaction and euphoria that accompany the successful completion of any task would be incomplete without the mention of the people who made it possible, whose constant guidance and encouragement crowned our effort with success. \newline
		 
		We express our sincere gratitude to the University Visvesvaraya College of Engineering, Bangalore, for having given us the opportunity to carry out this project. We would also like to thank our Principal, Dr. H. N. Ramesh, for providing us all the facilities to work on this project. \newline
	 
		We wish to place on record our grateful thanks to Dr. Dilip Kumar, Head of the Department, Department of Computer Science and Engineering, for providing us a wonderful opportunity to learn and the needful resources for this project. \newline 
	 
		We consider it a privilege and an honour to express our sincere gratitude to our guide, Mrs. P. Deepa Shenoy, for her valuable guidance throughout the tenure of this project work, and whose support and encouragement made this work possible.  \newline
	 
		We also thank our parents and our friends for their help, encouragement and support. Last but not the least; we thank God Almighty, without whose blessings this wouldn't have been possible. \newline 

		\hfill \textbf{Thanking You} 

		\hfill Jerry Allan Akshay 

		\hfill Shiva Pundir 

		\hfill Rahul R. 

		\hfill Sai Kumar 

	%Abstract

	\chapter*{Abstract}\label{chapter}
		
	\lhead{\titleref{chapter}}

		The project 'Universal Interpreter' was built as an application to help the differently-abled people, i.e. deaf, dumb, blind or combination of any, by using Image Recognition and AI/ML. \newline
		
		The present world we live in, a world dominated by visual and audio peripherals, can prove to be a tough place to live in for differently abled people. Hence, this project was inspired by the need to use the very same dominating technologies to help the differently abled overcome their challenges. \newline 

		With Universal Interpreter, we make use of Morse Codes and the American Sign Language - two of the standardised forms of communication all over the world - to ensure that any differently-abled person can use the application for its intended purposes.\newline 

	

	%Table of Contents

	\lhead{Table of Contents}

	\renewcommand{\footrulewidth}{0pt}

	\tableofcontents

	\newpage

	%List of Figures and Tables

	\listoffigures

	\vfill

	\listoftables

	\newpage

	\renewcommand{\footrulewidth}{0.4pt}

	\pagenumbering{arabic} %Use Page Numbering from here


	%Footer
	\fancyfoot[R]{\changefont Page \thepage}

	%Introduction

	\chapter{Introduction}\label{chapter1}
		
	\lhead{\titleref{chapter1}}
	
		% Redefine the plain page style
		\fancypagestyle{plain}{%
		\fancyhf{}%
		\fancyfoot[R]{Page \thepage}%
		\renewcommand{\headrulewidth}{0pt}% Line at the header invisible
		\renewcommand{\footrulewidth}{0.4pt}% Line at the footer visible
		}

		\section{Problem Definition}
			The world we live in is a world dominated by technology made for people with no impairments. Smartphones and Computers form a major portion of this technology. In the current age and time, most of the communication between people takes place through the use of smartphones and computers via SMS(s), Emails WhatsApp messages and so forth. \\
			
			As such, physically impaired people don't get to reap the benefits of smartphones, computers and other technologies catered to the non-impaired as there are no appropriate interfaces designed for use by them on these devices.
		
			\section{Objectives}
			The main objective of this project is to set up interfacing mechanisms for the physically-impaired - specifically the deaf, dumb and blind - and design an application for the purpose of communication between the users of the aforementioned application. The objectives can be postulated as follows:
			\begin{itemize}
				\item To set up interfacing mechanisms using Morse Codes, Text, Speech and the American Sign Language. 
				\item To design an Android Application for the purposes of communication using the aformentioned interfaces.
				\item To structure and use a no-SQL database in the appropriate format for the purpose of use of the Android Application.
				\item To train and use a Neural Network Model for the purposes of recognizing the Hand Gestures in the form of the American Sign Language through the use of a Camera.
			\end{itemize}
				\section{Standardised Forms of Communication}
				In general, communication is simply the act of transferring information from one place, person or group to another. Every communication involves (at least) one sender, a message and a recipient. This may sound simple, but communication is actually a very complex subject. The transmission of the message from sender to recipient can be affected by a huge range of things. These include our emotions, the cultural situation, the medium used to communicate, and even our location. The complexity is why good communication skills are considered so desirable by employers around the world: accurate, effective and unambiguous communication is actually extremely hard.\\

				With regards to communication, there are various internationally accepted and standardised forms of communication. These standardised forms imply the universal understanding of the information being conveyed with the help of the standardised medium.\\

				Since our project is aimed to help the physically impaired such as those who are deaf, dumb, blind, or a combination of these, we make use of two particular standardised forms of communication that are proven to be used effectively for communication by the physically impaired, namely Morse Codes and the American Sign Language (ASL). These two forms of communication are discussed in detail below. 
			\subsection{Morse Codes}

				\begin{figure}[h]
					\includegraphics[width=12cm]{Morse Code.png}
					\centering
					\caption{The International Morse Code}
				\end{figure}

				Morse code is a method used in telecommunication to encode text characters as standardized sequences of two different signal durations, called dots and dashes or dits and dahs. Morse code is named for Samuel Morse, an inventor of the telegraph.\\

				The International Morse Code encodes the 26 English letters A through Z, some non-English letters, the Arabic numerals 0 through 9 and a small set of punctuation and procedural signals (prosigns). There is no distinction between upper and lower case letters. Each Morse code symbol is formed by a sequence of dots and dashes. The dot duration is the basic unit of time measurement in Morse code transmission. The duration of a dash is three times the duration of a dot. Each dot or dash within a character is followed by a period of signal absence, called a space, equal to the dot duration. The letters of a word are separated by a space of duration equal to three dots, and the words are separated by a space equal to seven dots. To increase the efficiency of encoding, Morse code was designed so that the length of each symbol is approximately inverse to the frequency of occurrence in text of the English language character that it represents. Thus the most common letter in English, the letter "E", has the shortest code: a single dot. Because the Morse code elements are specified by proportion rather than specific time durations, the code is usually transmitted at the highest rate that the receiver is capable of decoding. The Morse code transmission rate (speed) is specified in groups per minute, commonly referred to as words per minute.\\
				
				Morse code is usually transmitted by on-off keying of an information-carrying medium such as electric current, radio waves, visible light, or sound waves. The current or wave is present during the time period of the dot or dash and absent during the time between dots and dashes. Morse code can be memorized, and Morse code signalling in a form perceptible to the human senses, such as sound waves or visible light, can be directly interpreted by persons trained in the skill.\\

				Because many non-English natural languages use other than the 26 Roman letters, Morse alphabets have been developed for those languages.\\

				\subsubsection{Applications for the General Public}

					\begin{figure}[h]
						\includegraphics[width=7cm]{SOS - Morse.png}
						\centering
						\caption{Representation of the SOS Morse Code}
					\end{figure}

					An important application is signalling for help through SOS, "\textbf{. . . \_ \_ \_ . . .}". This can be sent many ways: keying a radio on and off, flashing a mirror, toggling a flashlight, and similar methods. SOS is not three separate characters, rather, it is a prosign SOS, and is keyed without gaps between characters.\\
					
					Some Nokia mobile phones offer an option to alert the user of an incoming text message with the Morse tone "\textbf{. . . \_ \_ . . .}" (representing SMS or Short Message Service). In addition, applications are now available for mobile phones that enable short messages to be input in Morse Code.
				
					\subsubsection{Morse Code as an Assistive Technology}
						Morse code has been employed as an assistive technology, helping people with a variety of disabilities to communicate. For example, the Android operating system versions 5.0 and higher allow users to input text using Morse Code as an alternative to a keypad or handwriting recognition.\\

						Morse can be sent by persons with severe motion disabilities, as long as they have some minimal motor control. An original solution to the problem that caretakers have to learn to decode has been an electronic typewriter with the codes written on the keys. Codes were sung by users; see the voice typewriter employing morse or votem, Newell and Nabarro, 1968.\\

						Morse code can also be translated by computer and used in a speaking communication aid. In some cases, this means alternately blowing into and sucking on a plastic tube ("sip\-and\-puff" interface). An important advantage of Morse code over row column scanning is that once learned, it does not require looking at a display. Also, it appears faster than scanning.\\

						In one case reported in the radio amateur magazine \textit{QST}, an old shipboard radio operator who had a stroke and lost the ability to speak or write could communicate with his physician (a radio amateur) by blinking his eyes in Morse. Two examples of communication in intensive care units were also published in \textit{QST}, Another example occurred in 1966 when prisoner of war Jeremiah Denton, brought on television by his North Vietnamese captors, Morse-blinked the word \textit{TORTURE}. In these two cases, interpreters were available to understand those series of eye-blinks.
					
			\subsection{American Sign Language}
				The American Sign Language (ASL) is a natural language i.e., a natural language or ordinary language is any language that has evolved naturally in humans through use and repetition without conscious planning or premeditation, that serves as the predominant sign language of Deaf communities in the United States and most of Anglophone Canada. Besides North America, dialects of ASL and ASL-based creoles are used in many countries around the world, including much of West Africa and parts of Southeast Asia. ASL is also widely learned as a second language, serving as a lingua franca. ASL is most closely related to French Sign Language (LSF). It has been proposed that ASL is a creole language of LSF, although ASL shows features atypical of creole languages, such as agglutinative morphology.\\

				ASL originated in the early 19th century in the American School for the Deaf (ASD) in West Hartford, Connecticut, from a situation of language contact. Since then, ASL use has propagated widely by schools for the deaf and Deaf community organizations. Despite its wide use, no accurate count of ASL users has been taken. Reliable estimates for American ASL users range from 250,000 to 500,000 persons, including a number of children of deaf adults. ASL users face stigma due to beliefs in the superiority of oral language to sign language.\\

				ASL signs have a number of phonemic components, such as movement of the face, the torso, and the hands. ASL is not a form of pantomime although iconicity plays a larger role in ASL than in spoken languages. English loan words are often borrowed through fingerspelling, although ASL grammar is unrelated to that of English. ASL has verbal agreement and aspectual marking and has a productive system of forming agglutinative classifiers. Many linguists believe ASL to be a subject–verb–object (SVO) language. However, there are several alternative proposals to account for ASL word order.
			
				\begin{figure}[h]
					\includegraphics[width=12cm]{ASL Signs.png}
					\centering
					\caption{ASL Hand Signs}
				\end{figure}

				\subsubsection{Finger-Spelling}
					ASL possesses a set of 26 signs known as the American manual alphabet, which can be used to spell out words from the English language. Such signs make use of the 19 handshapes of ASL.  example, the signs for `p' and `k' use the same handshape but different orientations. A common misconception is that ASL consists only of fingerspelling; although such a method (Rochester Method) has been used, it is not ASL.\\

					Fingerspelling is a form of borrowing, a linguistic process wherein words from one language are incorporated into another. In ASL, fingerspelling is used for proper nouns and for technical terms with no native ASL equivalent. There are also some other loan words which are fingerspelled, either very short English words or abbreviations of longer English words, e.g. O-N from English `on', and A-P-T from English `apartment'. Fingerspelling may also be used to emphasize a word that would normally be signed otherwise.\\
				
		\section{Tools}
			There are various tools that have been used to realize the project Universal Interpreter. The combinatorial usage of these tools has helped to fabricate this project. As such, there are two categories of tools - The Front-end Tools and the Back-end Tools. \\
			
			Front-end and Back-end are terms used by programmers and computer professionals to describe the layers that make up hardware, a computer program or a website which are delineated based on how accessible they are to a user. These two categories of tools are further discussed below. 
			\subsection{Front-end Tools}
				The layer above the back end is the front end and it includes all software or hardware that is part of a user interface. Human or digital users interact directly with various aspects of the front end of a program, including user-entered data, buttons, programs, websites and other features. Most of these features are designed by user experience (UX) professionals to be accessible, pleasant and easy to use. The Front-end tools used for this project are mentioned below.
				\subsubsection{Android Studio}
				Android Studio is the official integrated development environment (IDE) for Google's Android operating system, built on JetBrains' IntelliJ IDEA software and designed specifically for Android development. It is available for download on Windows, macOS and Linux based operating systems. It is a replacement for the Eclipse Android Development Tools (ADT) as the primary IDE for native Android application development.\\

				Android Studio was announced on May 16, 2013 at the Google I/O conference. It was in early access preview stage starting from version 0.1 in May 2013, then entered beta stage starting from version 0.8 which was released in June 2014. The first stable build was released in December 2014, starting from version 1.0.\\
				
				On May 7, 2019, Kotlin replaced Java as Google’s preferred language for Android app development. Java is still supported, as is C++.\\ 
				
				On top of IntelliJ's powerful code editor and developer tools, Android Studio offers even more features that enhance your productivity when building Android apps, such as:
					\begin{itemize}
						\item A flexible Gradle-based build system
						\item A fast and feature-rich emulator
						\item A unified environment where you can develop for all Android devices
						\item Apply Changes to push code and resource changes to your running app without restarting your app
						\item Code templates and GitHub integration to help you build common app features and import sample code
						\item Extensive testing tools and frameworks
						\item Lint tools to catch performance, usability, version compatibility, and other problems
						\item C++ and NDK support
						\item Built-in support for Google Cloud Platform, making it easy to integrate Google Cloud Messaging and App Engine
					\end{itemize}
				\subsubsection{Python}
					Python is an interpreted, high-level, general-purpose programming language. Created by Guido van Rossum and first released in 1991, Python's design philosophy emphasizes code readability with its notable use of significant whitespace. Its language constructs and object-oriented approach aim to help programmers write clear, logical code for small and large-scale projects.\\

					Python is dynamically typed and garbage-collected. It supports multiple programming paradigms, including procedural, object-oriented, and functional programming. Python is often described as a "batteries included" language due to its comprehensive standard library.\\

					Python was conceived in the late 1980s as a successor to the ABC language. Python 2.0, released in 2000, introduced features like list comprehensions and a garbage collection system capable of collecting reference cycles. Python 3.0, released in 2008, was a major revision of the language that is not completely backward-compatible, and much Python 2 code does not run unmodified on Python 3.\\

					The Python 2 language, i.e. Python 2.7.x, was officially discontinued on 1 January 2020 (first planned for 2015) after which security patches and other improvements will not be released for it. With Python 2's end-of-life, only Python 3.5.x and later are supported.\\

					Python interpreters are available for many operating systems. A global community of programmers develops and maintains CPython, an open source reference implementation. A non-profit organization, the Python Software Foundation, manages and directs resources for Python and CPython development.\\

					The use of Python can span through countless yarns of thread. Few of the many capabilities of Python are (but are not limited to):
					\begin{itemize}
						\item It can be used on a server to create web applications.
						\item It can be used alongside software to create workflows.
						\item It can connect to database systems. It can also read and modify files.
						\item It can be used to handle big data and perform complex mathematics.
						\item It can be used for rapid prototyping, or for production-ready software development.
					\end{itemize}

					With the use of certain framworks and libraries, Python can very well be used to develop certain front-end interfaces.
			\subsection{Back-end Tools}
		
				The back end refers to parts of a computer application or a program's code that allow it to operate and that cannot be accessed by a user. Most data and operating syntax are stored and accessed in the back end of a computer system. Typically the code is comprised of one or more programming languages. The back end is also called the data access layer of software or hardware and includes any functionality that needs to be accessed and navigated to by digital means.\\

				A back-end application or program supports front-end user services, and interfaces with any required resources. The back-end application may interact directly with the front end or it may be called from an intermediate program that mediates front-end and back-end activities. Few of the back-end tools used in this project are discussed below.

				\subsubsection{Firebase}
					Firebase is a mobile and web application development platform developed by Firebase, Inc. in 2011, then acquired by Google in 2014. As of October 2018, the Firebase platform has 18 products, which are used by 1.5 million apps.\\
				
					Firebase evolved from Envolve, a prior startup founded by James Tamplin and Andrew Lee in 2011. Envolve provided developers an API that enables the integration of online chat functionality into their websites. After releasing the chat service, Tamplin and Lee found that it was being used to pass application data that were not chat messages. Developers were using Envolve to sync application data such as game state in real time across their users. Tamplin and Lee decided to separate the chat system and the real-time architecture that powered it. They founded Firebase as a separate company in September 2011 and it launched to the public in April 2012.\\

					Firebase's first product was the Firebase Real-time Database, an API that synchronizes application data across iOS, Android, and Web devices, and stores it on Firebase's cloud. The product assists software developers in building real-time, collaborative applications.\\

					In May 2012, a month after the beta launch, Firebase raised \$1.1 million in seed funding from venture capitalists Flybridge Capital Partners, Greylock Partners, Founder Collective, and New Enterprise Associates. In June 2013, the company further raised \$5.6 million in Series A funding from Union Square Ventures and Flybridge Capital Partners.\\

					In 2014, Firebase launched two products. Firebase Hosting and Firebase Authentication. This positioned the company as a mobile backend as a service.\\

					In October 2014, Firebase was acquired by Google. A year later, in October 2015, Google acquired Divshot, an HTML5 web-hosting platform, to merge it with the Firebase team.\\

					In May 2016, at Google I/O, the company's annual developer conference, Firebase introduced Firebase Analytics and announced that it was expanding its services to become a unified backend-as-a-service (BaaS) platform for mobile developers. Firebase now integrates with various other Google services, including Google Cloud Platform, AdMob, and Google Ads to offer broader products and scale for developers. Google Cloud Messaging, the Google service to send push notifications to Android devices, was superseded by a Firebase product, Firebase Cloud Messaging, which added the functionality to deliver push notifications to both iOS and web devices. In January 2017, Google acquired Fabric and Crashlytics from Twitter to add those services to Firebase.\\

					In October 2017, Firebase has launched Cloud Firestore, a real-time document database as the successor product to the original Firebase Realtime Database.
				\subsubsection{Neural Network Model}

					A neural network is a network or circuit of neurons, or in a modern sense, an artificial neural network, composed of artificial neurons or nodes. Thus a neural network is either a biological neural network, made up of real biological neurons, or an artificial neural network, for solving artificial intelligence (AI) problems. The connections of the biological neuron are modeled as weights. A positive weight reflects an excitatory connection, while negative values mean inhibitory connections.\\
					 
					All inputs are modified by a weight and summed. This activity is referred as a linear combination. Finally, an activation function controls the amplitude of the output. For example, an acceptable range of output is usually between 0 and 1, or it could be -1 and 1.\\

					These artificial networks may be used for predictive modeling, adaptive control and applications where they can be trained via a dataset. Self-learning resulting from experience can occur within networks, which can derive conclusions from a complex and seemingly unrelated set of information.\\

					For this particular project, we are using the Inception v3 Neural Network Model. Inception v3 is a convolutional neural network for assisting in image analysis and object detection, and got its start as a module for Googlenet. It is the third edition of Google's Inception Convolutional Neural Network, originally introduced during the ImageNet Recognition Challenge. Just as ImageNet can be thought of as a database of classified visual objects, Inception helps classification of objects in the world of computer vision. One such use is in life sciences, where it aids in the research of Leukemia. It was "codenamed 'Inception' after the film of the same name".\\

					In our particular use-case scenario, we use to Inception v3 Model to recognize the Hand Symbols for the different alphabets present in the American Sign Language (ASL).

				\subsubsection{Python}
					As discussed in the previous subsection on front-end tools, Python can very well be used for backend purposes too. With the use of the right APIs (Application Programming Interfaces), Python can be used for a wide variety of tasks such as server side programming, machine learning predictions, and many more.
				
				\subsubsection{Java}
					Java is a general-purpose programming language that is class-based, object-oriented, and designed to have as few implementation dependencies as possible. It is intended to let application developers write once, run anywhere (WORA), meaning that compiled Java code can run on all platforms that support Java without the need for recompilation. Java applications are typically compiled to bytecode that can run on any Java virtual machine (JVM) regardless of the underlying computer architecture. The syntax of Java is similar to C and C++, but it has fewer low-level facilities than either of them. As of 2019, Java was one of the most popular programming languages in use according to GitHub, particularly for client-server web applications, with a reported 9 million developers.\\

					Java was originally developed by James Gosling at Sun Microsystems (which has since been acquired by Oracle) and released in 1995 as a core component of Sun Microsystems' Java platform. The original and reference implementation Java compilers, virtual machines, and class libraries were originally released by Sun under proprietary licenses. As of May 2007, in compliance with the specifications of the Java Community Process, Sun had relicensed most of its Java technologies under the GNU General Public License. Meanwhile, others have developed alternative implementations of these Sun technologies, such as the GNU Compiler for Java (bytecode compiler), GNU Classpath (standard libraries), and IcedTea-Web (browser plugin for applets)\\

					In our particular project, we use Java for the design of the android application as it is one of the accepted languages to code in Android Studio.
	\newpage

	%System Requirements

	\chapter{System Requirements}\label{chapter2}
		
	\lhead{\titleref{chapter2}}
	
	
		\section{Software Requirements}
					\begin{enumerate}
						\item Database : Firebase
						\item Server Side Technology : Google FCM
						\item Language : Java, Python
						\item Client Side Technology : XML
						\item Cloud : Firebase Cloud
					\end{enumerate}
		\section{Hardware Requirements}
					\begin{enumerate}
						\item Processor : Mediatek Helio A22
						\item Operating System : Android 4.0 and above
						\item RAM : 3 GigaBytes
						\item Server : Firebase 
					\end{enumerate}
	\newpage

	%System Design

	\chapter{System Design}\label{chapter3}
		
	\lhead{\titleref{chapter3}}
	

		\section{Android Application}
			Android applications can be written using Kotlin, Java, and C++ languages. The Android SDK tools compile your code along with any data and resource files into an APK, an Android package, which is an archive file with an .apk suffix. One APK file contains all the contents of an Android app and is the file that Android-powered devices use to install the app.\\

			Each Android app lives in its own security sandbox, protected by the following Android security features:
			\begin{itemize}
				\item The Android operating system is a multi-user Linux system in which each app is a different user.
				\item By default, the system assigns each app a unique Linux user ID (the ID is used only by the system and is unknown to the app). The system sets permissions for all the files in an app so that only the user ID assigned to that app can access them.
				\item Each process has its own virtual machine (VM), so an app's code runs in isolation from other apps.
				\item By default, every app runs in its own Linux process. The Android system starts the process when any of the app's components need to be executed, and then shuts down the process when it's no longer needed or when the system must recover memory for other apps.

			\end{itemize}

			The Android system implements the principle of least privilege. That is, each app, by default, has access only to the components that it requires to do its work and no more. This creates a very secure environment in which an app cannot access parts of the system for which it is not given permission. However, there are ways for an app to share data with other apps and for an app to access system services:

			\begin{itemize}
				\item It's possible to arrange for two apps to share the same Linux user ID, in which case they are able to access each other's files. To conserve system resources, apps with the same user ID can also arrange to run in the same Linux process and share the same VM. The apps must also be signed with the same certificate.
				\item An app can request permission to access device data such as the device's location, camera, and Bluetooth connection. The user has to explicitly grant these permissions.
			\end{itemize}
			\subsection{Application Components}

				App components are the essential building blocks of an Android app. Each component is an entry point through which the system or a user can enter your app. Some components depend on others.\\
				
				There are four different types of app components:
				\begin{itemize}
					\item Activities
					\item Services
					\item Broadcast receivers
					\item Content providers
				\end{itemize}		

				\subsubsection{Activities}

					\begin{figure}[h]
						\includegraphics[width=12cm]{App Life Cycle.png}
						\centering
						\caption{Android Application Life Cycle}
					\end{figure}

					An activity is the entry point for interacting with the user. It represents a single screen with a user interface. For example, an email app might have one activity that shows a list of new emails, another activity to compose an email, and another activity for reading emails. Although the activities work together to form a cohesive user experience in the email app, each one is independent of the others. As such, a different app can start any one of these activities if the email app allows it. For example, a camera app can start the activity in the email app that composes new mail to allow the user to share a picture. An activity facilitates the following key interactions between system and app:
					\begin{itemize}
						\item Keeping track of what the user currently cares about (what is on screen) to ensure that the system keeps running the process that is hosting the activity.
						\item Knowing that previously used processes contain things the user may return to (stopped activities), and thus more highly prioritize keeping those processes around.
						\item Helping the app handle having its process killed so the user can return to activities with their previous state restored.
						\item Providing a way for apps to implement user flows between each other, and for the system to coordinate these flows. (The most classic example here being share.)
					\end{itemize}

					You implement an activity as a subclass of the Activity class.
				\subsubsection{Services}
					A service is a general-purpose entry point for keeping an app running in the background for all kinds of reasons. It is a component that runs in the background to perform long-running operations or to perform work for remote processes. A service does not provide a user interface. For example, a service might play music in the background while the user is in a different app, or it might fetch data over the network without blocking user interaction with an activity. Another component, such as an activity, can start the service and let it run or bind to it in order to interact with it. There are actually two very distinct semantics services tell the system about how to manage an app: Started services tell the system to keep them running until their work is completed. This could be to sync some data in the background or play music even after the user leaves the app. Syncing data in the background or playing music also represent two different types of started services that modify how the system handles them:
					\begin{itemize}
						\item Music playback is something the user is directly aware of, so the app tells the system this by saying it wants to be foreground with a notification to tell the user about it; in this case the system knows that it should try really hard to keep that service's process running, because the user will be unhappy if it goes away.
						\item A regular background service is not something the user is directly aware as running, so the system has more freedom in managing its process. It may allow it to be killed (and then restarting the service sometime later) if it needs RAM for things that are of more immediate concern to the user.
					\end{itemize}	

					Bound services run because some other app (or the system) has said that it wants to make use of the service. This is basically the service providing an API to another process. The system thus knows there is a dependency between these processes, so if process A is bound to a service in process B, it knows that it needs to keep process B (and its service) running for A. Further, if process A is something the user cares about, then it also knows to treat process B as something the user also cares about. Because of their flexibility (for better or worse), services have turned out to be a really useful building block for all kinds of higher-level system concepts. Live wallpapers, notification listeners, screen savers, input methods, accessibility services, and many other core system features are all built as services that applications implement and the system binds to when they should be running.\\

					A service is implemented as a subclass of Service. 
				\subsubsection{Broadcast Receivers}
					A broadcast receiver is a component that enables the system to deliver events to the app outside of a regular user flow, allowing the app to respond to system-wide broadcast announcements. Because broadcast receivers are another well-defined entry into the app, the system can deliver broadcasts even to apps that aren't currently running. So, for example, an app can schedule an alarm to post a notification to tell the user about an upcoming event... and by delivering that alarm to a BroadcastReceiver of the app, there is no need for the app to remain running until the alarm goes off. Many broadcasts originate from the system—for example, a broadcast announcing that the screen has turned off, the battery is low, or a picture was captured. Apps can also initiate broadcasts—for example, to let other apps know that some data has been downloaded to the device and is available for them to use. Although broadcast receivers don't display a user interface, they may create a status bar notification to alert the user when a broadcast event occurs. More commonly, though, a broadcast receiver is just a gateway to other components and is intended to do a very minimal amount of work. For instance, it might schedule a JobService to perform some work based on the event with JobScheduler\\
					
					A broadcast receiver is implemented as a subclass of BroadcastReceiver and each broadcast is delivered as an Intent object.	
				\subsubsection{Content Providers}
					A content provider manages a shared set of app data that you can store in the file system, in a SQLite database, on the web, or on any other persistent storage location that your app can access. Through the content provider, other apps can query or modify the data if the content provider allows it. For example, the Android system provides a content provider that manages the user's contact information. As such, any app with the proper permissions can query the content provider, such as ContactsContract.Data, to read and write information about a particular person. It is tempting to think of a content provider as an abstraction on a database, because there is a lot of API and support built in to them for that common case. However, they have a different core purpose from a system-design perspective. To the system, a content provider is an entry point into an app for publishing named data items, identified by a URI scheme. Thus an app can decide how it wants to map the data it contains to a URI namespace, handing out those URIs to other entities which can in turn use them to access the data. There are a few particular things this allows the system to do in managing an app:
					\begin{itemize}
						\item Assigning a URI doesn't require that the app remain running, so URIs can persist after their owning apps have exited. The system only needs to make sure that an owning app is still running when it has to retrieve the app's data from the corresponding URI.
						\item These URIs also provide an important fine-grained security model. For example, an app can place the URI for an image it has on the clipboard, but leave its content provider locked up so that other apps cannot freely access it. When a second app attempts to access that URI on the clipboard, the system can allow that app to access the data via a temporary URI permission grant so that it is allowed to access the data only behind that URI, but nothing else in the second app.
					\end{itemize}

					Content providers are also useful for reading and writing data that is private to your app and not shared.\\

					A content provider is implemented as a subclass of ContentProvider and must implement a standard set of APIs that enable other apps to perform transactions.\newline \newline

					A unique aspect of the Android system design is that any app can start another app’s component. For example, if you want the user to capture a photo with the device camera, there's probably another app that does that and your app can use it instead of developing an activity to capture a photo yourself. You don't need to incorporate or even link to the code from the camera app. Instead, you can simply start the activity in the camera app that captures a photo. When complete, the photo is even returned to your app so you can use it. To the user, it seems as if the camera is actually a part of your app.\\

					When the system starts a component, it starts the process for that app if it's not already running and instantiates the classes needed for the component. For example, if your app starts the activity in the camera app that captures a photo, that activity runs in the process that belongs to the camera app, not in your app's process. Therefore, unlike apps on most other systems, Android apps don't have a single entry point (there's no main() function).\\

					Because the system runs each app in a separate process with file permissions that restrict access to other apps, your app cannot directly activate a component from another app. However, the Android system can. To activate a component in another app, deliver a message to the system that specifies your intent to start a particular component. The system then activates the component for you.
			\subsection{Front-end Features}
					The front-end of an application mainly deals with the User-Interface (UI) of the application. The app's user interface is everything that the user can see and interact with. Android provides a variety of pre-built UI components such as structured layout objects and UI controls that allow you to build the graphical user interface for your app. Android also provides other UI modules for special interfaces such as dialogs, notifications, and menus.
					\subsubsection{Layouts}
					\begin{figure}[b]
						\includegraphics[width=12cm]{ViewGroup.png}
						\centering
						\caption{Illustration of a view hierarchy, which defines a UI layout}
					\end{figure}
					A layout defines the structure for a user interface in your app, such as in an activity. All elements in the layout are built using a hierarchy of View and ViewGroup objects. A View usually draws something the user can see and interact with. Whereas a ViewGroup is an invisible container that defines the layout structure for View and other ViewGroup objects, as shown in Figure 3.2.\\

					The View objects are usually called "widgets" and can be one of many subclasses, such as Button or TextView. The ViewGroup objects are usually called "layouts" can be one of many types that provide a different layout structure, such as LinearLayout or ConstraintLayout.\\

					You can declare a layout in two ways:
					\begin{itemize}
						\item \textbf{Declare UI elements in XML.} Android provides a straightforward XML vocabulary that corresponds to the View classes and subclasses, such as those for widgets and layouts. You can also use Android Studio's Layout Editor to build your XML layout using a drag-and-drop interface.
						\item \textbf{Instantiate layout elements at runtime.} Your app can create View and ViewGroup objects (and manipulate their properties) programmatically.
					\end{itemize}

					Declaring your UI in XML allows you to separate the presentation of your app from the code that controls its behavior. Using XML files also makes it easy to provide different layouts for different screen sizes and orientations.\\

					The Android framework gives you the flexibility to use either or both of these methods to build your app's UI. For example, you can declare your app's default layouts in XML, and then modify the layout at runtime.

				\subsubsection{Notification Overviews}
					A notification is a message that Android displays outside your app's UI to provide the user with reminders, communication from other people, or other timely information from your app. Users can tap the notification to open your app or take an action directly from the notification.\\

					Notifications appear to users in different locations and formats, such as an icon in the status bar, a more detailed entry in the notification drawer, as a badge on the app's icon, and on paired wearables automatically. When you issue a notification, it first appears as an icon in the status bar. Users can swipe down on the status bar to open the notification drawer, where they can view more details and take actions with the notification. Users can drag down on a notification in the drawer to reveal the expanded view, which shows additional content and action buttons, if provided. A notification remains visible in the notification drawer until dismissed by the app or the user.
			
				\subsubsection{Dialogs}
					A dialog is a small window that prompts the user to make a decision or enter additional information. A dialog does not fill the screen and is normally used for modal events that require users to take an action before they can proceed. The Dialog class is the base class for dialogs, but you should avoid instantiating Dialog directly. Instead, use one of the following subclasses:
					\begin{itemize}
						\item \textbf{AlertDialog:} A dialog that can show a title, up to three buttons, a list of selectable items, or a custom layout.
						\item \textbf{DatePickerDialog} or \textbf{TimePickerDialog:} A dialog with a pre-defined UI that allows the user to select a date or time.
					\end{itemize}
					
					These classes define the style and structure for your dialog, but you should use a DialogFragment as a container for your dialog. The DialogFragment class provides all the controls you need to create your dialog and manage its appearance, instead of calling methods on the Dialog object.\\
					
					Using DialogFragment to manage the dialog ensures that it correctly handles lifecycle events such as when the user presses the Back button or rotates the screen. The DialogFragment class also allows you to reuse the dialog's UI as an embeddable component in a larger UI, just like a traditional Fragment (such as when you want the dialog UI to appear differently on large and small screens).
			
				\subsubsection{Menus}
					Menus are a common user interface component in many types of applications. To provide a familiar and consistent user experience, you should use the Menu APIs to present user actions and other options in your activities.

				\subsubsection{Toasts}
					A toast provides simple feedback about an operation in a small popup. It only fills the amount of space required for the message and the current activity remains visible and interactive. Toasts automatically disappear after a timeout.
			\subsection{Back-end Features}
			\subsection{Integration of Front-end and Back-end}
		\section{Inception v3 Neural Network}
		\section{Application Programming Interfaces}
			\subsection{TensorFlow Lite API}
			\subsection{Firebase API}
			\subsection{Google Text-to-Speech API}
			\subsection{Google Speech-to-Text API}

	\newpage

	%Project Implementation

	\chapter{Project Implementation}\label{chapter4}
		
	\lhead{\titleref{chapter4}}
	

		\section{Project Flow}
			\subsection{Phase I}
			\subsection{Phase II}
			\subsection{Phase III}

		\section{Android Application Implementation}
			\subsection{Design Ideologies}
			\subsection{Application Phases}
			\subsection{Database Structure}
		
		\section{Inception v3 Neural Network}
			\subsection{Data Collection}
			\subsection{Model Training}
			\subsection{Model Predictions}


	\newpage

	%Project Snapshots

	\chapter{Project Snapshots}\label{chapter5}
		
	\lhead{\titleref{chapter5}}
	

		\section{Landing Page}
		\section{Settings/Preferences Page}
		\section{Input Methods}

	\newpage

	%Conclusion

	\chapter{Conclusion}\label{chapter6}
		
	\lhead{\titleref{chapter6}}
	

	\newpage

	%References

	\chapter*{References}\label{chapter7}
		
	\lhead{\titleref{chapter7}}
	

	\newpage


\end{document} %Document Ends